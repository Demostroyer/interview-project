\documentclass[10pt,twocolumn]{IEEEtran}

\usepackage[linesnumbered,lined,boxed,commentsnumbered]{algorithm2e}

\title{Analyzing Algorithmic Patterns Based on Real Coding Interview Questions}
\author{Ian Dempsey}


\begin{document}

\maketitle
\pagenumbering{gobble}
\newpage
\pagenumbering{arabic}

\section{General Patterns for Analysis}
\subsection{Palindrome}
The following description is a general algorithm for solving the palindrome problem which is a common problem in Java and other languages. This approach can be used to solve numerous other problems by altering the inside of the loop.
\IncMargin{1em}
\begin{algorithm}
	\SetAlgoLined
	\KwData{Given input of characters, S}
	\KwResult{Boolean}
	initialization\;
	$leftIndex  \longleftarrow $S[0]\;
	$rightIndex \longleftarrow $S.length-1\;
	\While{leftIndex < rightIndex}{
	compare leftIndex with rightIndex\;
	\eIf{leftIndex !=rightIndex}{
		return false\;
	}
	 leftIndex++\;
	rightIndex\;
}
return true\;
\caption{The Palindrome Algorithm}
\end{algorithm}\DecMargin{1em}

\subsection{Merge Sort}
Merge Sort uses the idea of divide and conquer, this means the list to be sorted should be divided up into equal parts first, then these new smaller parts should be sorted individually first before recreating the full list.
Pseudocode:
\IncMargin{1em}
\begin{algorithm}
	\SetAlgoLined
	\KwData{List of unsorted data}
	\KwResult{Sorted List}
 	\eIf{ length of A is 1}{ return 1}
  	{Split A into two halves , L and R. Repeat until size of part =1\\
  	Sort each part individually \\
  	Merge with another subdivided section into B, the sorted list\\ 
  	Return B, the sorted structure}
\caption{The Merge Sort Algorithm through Recursion}
\end{algorithm}\DecMargin{1em}





\end{document}