\documentclass{article}
\usepackage[hidelinks]{hyperref}
\usepackage{graphicx} 

\title{Interim Report}
\author{Ian Dempsey}

\begin{document}
\begin{titlepage}
\maketitle
\begin{figure}[h]
\includegraphics{MU.png}
\end{figure} 
\end{titlepage}
\pagenumbering{arabic}
\section{Goals of the Project}
The goal of this project is to create a report on the similarities between interview questions and the standard algorithmic patterns. These algorithms are ones which are commonly asked by interviewers. We plan to gather some information on these questions from online resources which discuss the experiences of users in interviews, and then create a list of important algorithms to focus on. We will then attempt to solve numerous questions which are asked in interviews and relate them to the previously selected patterns. In this project we will be comparing each of the similar questions to the standard patterns and highlighting how they are similar. We will be showing the reader that if they learn the core algorithms and then see how they can be slightly altered to perform differently, they can solve a plethora of questions.

\section{Overview of Background}
\par Nowadays there are a lot of online repositories for interview questions and interview related material in general. Websites such as hackerrank.com , glassdoor.ie, geeksforgeeks.org ,and leetcode.com all offer information on what questions are common during interviews for different companies. They also offer ways to test potential solutions to some of the given questions. 
\par For this paper we chose to use leetcode.com \cite{leet1}, as this website comes with a huge amount of online material that is generally taken from previously asked interview questions. Companies also use sites just like leetcode when they are building up a question bank to ask candidates. This was important in our decision for the repository, as it meant we were using a platform which was regularly updated. LeetCode itself also has some built-in features which really appealed to us. The website has an online discussion forum for each question, allowing the community to discuss solutions, issues and the questions themselves. LeetCode also host their own weekly coding competitions, which allows users to get even more experience and confidence in coding problems. One of the main features which this online judge website has in comparison to the others is a section for a user to perform a mock interview. This mock interview is under the time constraint of a normal real world interview, this was a big attraction for us in choosing our online judge. The online editor that is used by leetcode also allows a user to select from a multitude of programming languages. For the purpose of this paper the language that was chosen was java. The reason for this is because it is a language most people learn first.
\section{Progress to Date}
We have chosen to focus on the following algorithms:
\begin{itemize}
\item Sorting Algorithms
\item Searching Algorithms
\item Graphs
\item Trees
\item Dynamic Programming
\end{itemize}
Inside each of these we have chosen to look at specific patterns in them, for example in sorting algorithms we have chosen to look at mergesort, and in searching we have employed the palindrome algorithm. 
So far we have completed 25, ranging from sorting algorithms to trees in the above list. We plan to solve around 40 questions by the submission date of this project. We have yet to study dynamic programming, and we would like to try and solve another pattern in the sorting algorithms, namely quicksort. 
\par The progress of the thesis is good. We have completed approximately half of it, and have planned out how to complete the write-up when we have completed more questions and conclude our research for the evaluation and consclusion of this paper.
\par All of the questions solved so far, and the thesis are available on my github \cite{gitHub}.  
\section{Problems Encountered}
The only major issue so far worth noting is the problem descriptions on leetcode. Some of them can be quite ambiguous, and even with the example they give it can still be confusing. This has caused us to take longer in solving a question than anticipated. 
Learning LATEX was also hard at first, but now is not a problem.
  
\section{Planned Next Steps}
We plan to solve around 15 more questions on leetcode. These questions will be focused in the dynamic programming area and possibly quicksort. We will be solving more questions in the areas we have previously worked on. 
With regards to the thesis, we will just keep working on it as we progress with more and more questions. 

\bibliography{mybib}
\bibliographystyle{unsrt}
\end{document}